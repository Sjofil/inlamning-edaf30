\documentclass[12pt]{article}
\usepackage[utf8]{inputenc}
\usepackage{fancyhdr}
\usepackage{graphicx}
\usepackage{geometry}
\usepackage{float}
\usepackage[utf8]{inputenc}
\usepackage{authblk}
 
\usepackage{array}
\usepackage{makecell}

% ---- Commands ------- %
\newcommand{\documentNumber}[1]{
    \LARGE  \textbf{Inlämningsuppgift}
    \\
    \medskip
}
\graphicspath{ {./images/} }
\newcommand{\documentVersion}[1]{
    \medskip
}
\newcommand{\documentTitle}[1]{
    \centerline{\rule{13cm}{0.4pt}}
    \bigskip \bigskip
    \LARGE \textbf{EDAF30} \\
    \bigskip
    \LARGE {#1} \\
    \bigskip \bigskip
    \centerline{\rule{13cm}{0.4pt}}
}

\newcommand{\documentDate}[1]{
    \date {#1} 
}

\renewcommand\theadalign{bc}
\renewcommand\theadfont{\bfseries}
\renewcommand\theadgape{\Gape[5pt]}
\renewcommand\cellgape{\Gape[5pt]}



\pagestyle{fancy}
\lhead{\leftmark}
\rhead{}
\rfoot{\thepage}
\cfoot{}
\lfoot{}


% ------------------------------------------------ #

% ----- FILL THIS ----- %
\title {
    \documentNumber {01}    

    % Full name - SHORTNAME
    \documentTitle {Dijkstras algoritm}
    
  
    
\author{Filip Sjövall - Oscar Blixt}



}

%\author{
% Filip Sjövall\\
%  \and
 % Oscar Blixt\\
%}






\begin{document}
\addtocontents{toc}{\protect\setcounter{tocdepth}{2}}
\maketitle
%%\end{document}
\thispagestyle{empty}


\newpage

\section*{Övergripande design}

Det inlämnade programmet innehåller fyra stycken klasser: Node, Egde, Nodeset samt Graph. Utöver dessa kommer programmet även med två stycken fira funktioner dijkstra samt generalDijsktra. Samtliga klasser och funtkioner kommer beskrivas nedan.

\subsection*{Node}
Denna klassen innehåller algoritemns noder. Noderna har ett namn(\textit{string}), ett värde(\textit{int}) samt en pekare till nodens förälder(\textit{Node}). Nodens värde är den totala kostanden det tar att gå från en startnod till den valda noden. Nodens förälder används för att spara undan information var noden befann sig i ett tidigare stadie i programmet. 
\begin{itemize}
    \item Hej
\end{itemize}


\subsection*{Edge}
Denna klassen innehåller algoritmens bågar. Bågarna har en längd(int) samt en destination(\textit{string}). Men hjälp av medlemsfunktionen \textit{addEdge} från \textit{Node} klassen kan man lägga till en båge som sträcker sig från en nod till en annan. Denna bågen används i algoritmen för att beräkna den totala kostnaden det tar att röra sig från en nod till en annan.

\subsection*{Nodeset}
Denna klassen ska innehåller en mängd av noder. Mängen har implementerats med en vektor av typen \textit{Node}. Med medlmensfunktionen removeMin kan den noden med lägst värde retuneras vilket också är vad funktionen används för i algoritmen, att hitta de närmast liggande noderna.

\subsection*{Graph}
Denna klassen innehåller funktion för att läsa in ett godtyckligt antal noder och bågar från en separat text fil med rader enligt: "Nod: kostnad destination". Denna information sparas i en vektor av typen \textit{Node}.

\subsection*{Dijkstra}
Funktionen används som en specifik lösning av dijkstras algoritm som hittar den kortaste vägen från en destination till en annan utifrån kostand från klassen \textit{Edge}.

\subsection*{generalDijkstra}
Funktionen används som en generaliserad lösning av dijkstras algoritm som hittar den kortaste vägen från en destination till en annan utifrån vad en...

\section*{Användarinstruktioner}
Källkoden finns paketerad i en zip-fil. När denna har har extraherats kan filerna länkas och kompileras med hjälp av kommandot \textit{make}. Efter detta kan programmer exekveras enligt av två valmöjligheter.
\begin{enumerate}
    \item Kommandot \textit{make premadeTest} kör endast färdigskrivna tester.
    \item Kommandot \textit{make ourTest} kör endast de tester som tagits fram utöver de färdigskrivna tester. 
\end{enumerate}


\section*{Brister och kommentarer}









\bibliographystyle{alpha}


\end{document}
